\documentclass[12pt,a4paper]{article}

\usepackage[T1]{fontenc}
\usepackage[utf8]{inputenc}
\usepackage[margin=2.5cm]{geometry}
\usepackage{graphicx}
\usepackage[hidelinks]{hyperref}
\usepackage{fancyhdr}
\usepackage{lastpage}
\usepackage{appendix}
\usepackage{color}
\usepackage{palatino}
\usepackage{changepage}
\usepackage{subcaption}
\usepackage{enumitem}
\usepackage{csquotes}
\usepackage{verbatim}
\usepackage{minted}
\usepackage[ruled,vlined]{algorithm2e}

\usemintedstyle{fruity}

\title{Binary Space Partitions}
\author{Carlos Requena López}

%% Fancy layout
\pagestyle{fancy}
\lhead{Binary Space Partitions}
\chead{}
\rhead{}
\lfoot{}
\cfoot{}
\rfoot{Page \thepage\ of \pageref{LastPage}}
\renewcommand{\headrulewidth}{0.4pt}
\renewcommand{\footrulewidth}{0.4pt}


%%% --- %%% --- DOCUMENT START --- %%% --- %%%
\begin{document}
\maketitle
\thispagestyle{empty}
\tableofcontents
\newpage
%%% Counting pages now %%%
\pagestyle{fancy}
\setcounter{page}{1}


\section{Introduction}

This first assignment deals with recursive autopartitioning of a two
dimensional space. Autopartitions, in contrast to partitions, are
performed extending the supporting lines of the segments already
present on the plane.

In particular, with partitions in general, we stop executing the
algorithm when there is

Having direct application in computer graphics, the most interesting
parameter when considering these partitions in the size of the
generated tree, since that's what is used in practice to decide what
to render first in the scene of a computer screen.

\section{Analysis and expectations}

We define the algorithm RandAutopart in \ref{algo}

\begin{algorithm}[h]
  \SetAlgoLined
  \KwIn{A set S of segments}
  \KwOut{The autopartition binary tree}

  \nl Pick a random permutation of S and take the first element\;
  \nl Create a node with this element in the tree.
\item
  \nl \While{there is a partition with at least 1 segment}{

  };
  \nl
\caption{\bf RandAutopart}
\label{algo}

\end{algorithm}

The upper bound found in literature \cite{Motwani:1995:RA:211390},
corresponds to the upper bound of the random algorithm for any
input. The expectation would be for this boundary to be closer when
choosing the worst possible input (segment positioning wise) and
further away when adding randomness to the input and even more when
giving the the most trivial input possible: parallel segments, in
which case the number of partitions would be $n+1$.

The algorithm will be therefore tested with a random input and a
crafted, ``evil'' input, to try and get close to this upper bound
without ever exceeding it.

\section{Implementation}

$n$ disjoint segments are generated and placed on a $H\times W$
canvas, where $H$ is the height and $W$ the width, both in pixels.

The first element of a random permutation of those $N$ segments is
taken, its supporting line extended, and the host space (the plane)
partitioned in two. The same procedure is applied recursively to

In order to know which segments another line intersects, some
primitive geometry operations have to be implemented. Since this is
done from scratch however, we expect them to not be as robust as the
ones some external library could provide. These operations include:

\begin{itemize}
\item Testing to which side a point is with respect to a line segment
  (of if it is on the line).
\item Testing whether a segment is on one side or the other of a
  segment (or if it is colinear).
\item Testing whether a segment could potentially intersect another if
  its supporting line is extended.
\item Testing whether two segments intersect each other.
\item Determining the coordinates of the intersection of a line and a
  segment.
\end{itemize}

Most operations rely of calculating the orientation determinant for
the line segment defined by two points and a third.

The generated segments' coordinates are integers to begin with and
Clojure uses a \texttt{Ratio} type to make arithmetic calculations as
exact as possible at the expense of performance, which is not an
issue.

Nonetheless, for the sake of simplicity and to avoid clutter, this
implementation is not exempt of errors that could arise due to
degenerate or special cases though.

\section{Results}

\section{Conclusion}

% \begin{figure}[ht!]
%   \centering
%   \includegraphics[width=1\textwidth]{some_graphic.png}
%   \caption{}
%   \label{fig:fig1}
% \end{figure}

\appendix
\section{Appendix - code listing}

\begin{minted}{clojure}
\end{minted}

\nocite{*}
\bibliographystyle{plain}
\bibliography{refs}

\end{document}
